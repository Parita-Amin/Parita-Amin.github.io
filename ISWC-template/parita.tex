% This is samplepaper.tex, a sample chapter demonstrating the
% LLNCS macro package for Springer Computer Science proceedings;
% Version 2.21 of 2022/01/12
%
\documentclass[runningheads]{llncs}
%
\usepackage[T1]{fontenc}
% T1 fonts will be used to generate the final print and online PDFs,
% so please use T1 fonts in your manuscript whenever possible.
% Other font encondings may result in incorrect characters.
%
\usepackage{graphicx}
% Used for displaying a sample figure. If possible, figure files should
% be included in EPS format.
%
% If you use the hyperref package, please uncomment the following two lines
% to display URLs in blue roman font according to Springer's eBook style:
%\usepackage{color}
%\renewcommand\UrlFont{\color{blue}\rmfamily}
%
\begin{document}
%
%\title{Creating a Linked Open Dataset for Feedback Survey Instrument\\}
%OR\\
\title{From Spreadsheet to Linked Open Data via Ontology}
%
%\titlerunning{ }
% If the paper title is too long for the running head, you can set
% an abbreviated paper title here
%
\author{Parita Amin\orcidID{0000-0003-3475-7223} \and
Daryl H. Hepting\orcidID{0000-0002-3138-3521}}
%
\authorrunning{P. Amin and D. H. Hepting}
% First names are abbreviated in the running head.
% If there are more than two authors, 'et al.' is used.
%
\institute{Department of Computer Science, University of Regina, Regina, SK S4S 0A2, Canada\\
\email{\{pda324,heptingd\}@uregina.ca}}
%
\maketitle              % typeset the header of the contribution
%
\begin{abstract}
The Semantic Web, over the years, has developed significantly in terms of study and research but still has numerous unanswered questions. Linked Open Vocabulary (LOV) for feedback survey is one of those topics which lack enough research content. This research focuses on development of Linked Open Data (LOD) using Protégé\footnote{\tt https://protegewiki.stanford.edu/wiki/Main\_Page} with an aim to convert data that is stored in tabular format (e.g. Spreadsheet, Comma Separated Values (CSV)) into LOD.

\keywords{Linked Open Data \and Survey \and Protégé \and Likert \and Open-ended.}

\end{abstract}

\section{Motivation}
Since 2012, Hepting has been collecting 
feedback\footnote{\tt http://www2.cs.uregina.ca/$\sim$hepting/teaching/evaluation.html} from students in the classes he teaches.
Most of the data collected has been on paper, but some recent course offerings have collected this feedback data using moodle's X feature. The feedback instrument has 9 Likert items using a 5 point Likert scale (Never, Rarely, Sometimes, Often, Always) and 3 open-ended questions. 
More detail on the feedback instrument can be found on his website\footnotemark[2].
A goal of making the dataset available as linked open data is to enable more robust 
analysis along with embracing
open data principles~\cite{principles}. 
Advice about publishing LOD can be found in works by Borgesius, Gray, and van Eechoud~\cite{guide} and W3C~\cite{advice}.

Presented here is the process to take this student feedback data from spreadsheet to 
fully linked open data, 
with the possibility of generalizing this effort to any sort of data that a 
citizen would like to publish.

The steps begin with searching for a Linked Open Vocabulary~\cite{gateway} that meets the citizen's needs.
If one is found, then that LOV can be used to map data from the spreadsheet into RDF. If one is not found, an ontology/vocabulary needs to be developed to allow the mapping to occur.

There are LOV's for surveys (see Scandolari et al.~\cite{ontology} and Fox and Katsumi~\cite{fox}) but they don't deal with Likert items.

The vocabulary need not be perfect the first time. The goal is to model the feedback instrument (responses). What is the most concise and most straightforward way to do this modelling? In SWftWO~\cite{swftwo}, the authors propose this test... 

\section{Development of Linked Open Dataset for Feedback Survey Using Protégé}
There exist many ontology editors for the development of LOVs, of which Protégé is one of the most used. ``Protégé is a free, open-source platform that provides a growing user community with a suite of tools to construct domain models and knowledge-based applications with ontologies''~\cite{protege}. 
Here, Protégé has been used to describe new vocabulary for student feedback survey. 
The vocabulary for feedback survey can be described in different ways. 
The vocabulary to describe the survey instrument (the form) 
would be different from the vocabulary used to describe the responses. 
This paper focuses on describing the vocabulary for the survey instrument which may consist of the ``Type of Survey'', ``Survey Details'', and ``Survey Questions''. 
These concepts form the basis for the survey instrument.

\subsection{Classes and Class Hierarchy}
The root node of the class hierarchy in Protégé is ``owl:Thing''. ``Survey'', ``SurveyQuestions'',  ``SurveyAnswers'', ``Respondent'', and ``ResponseDate'' are the sub-classes of ``owl:Thing''. 
The class hierarchy can be further built using Tools > Create class hierarchy... 
The ``Survey'' class focuses on the type of the survey. 
The ``Respondent'' class signifies the respondents of the survey and the ``ResponseDate'' is for the date on which the respondent responded to they survey. The ``SurveyQuestions'' class represents 
the question section whereas the ``SurveyAnswers'' class represents 
the answer section of the survey. ``FeedbackSurvey'' is the only subclass of the ``Survey'' class where as the ``SurveyQuestions'' and "SurveyAnswers" have two subclasses each, corresponding to likert and open-ended type question-answers.
The ``SubClass Of'' field of Class Description view allows describing the parent class(es)

\subsection{Object Properties and Object Property Hierarchy}
The next step in ontology development in Protégé is describing the object properties. 
Object properties are used to describe the relationship between two entities. 
For instance, ``Survey'' and ``FeedbackSurvey'' classes are the entities that can be connected to each other using an object property (relationship). ``Survey'' is connected to ``FeedbackSurvey'' by ``hasType'' object property and ``FeedbackSurvey'' is connected to ``Survey'' by ``isTypeOf''. In such cases, ``hasType'' is said to be an inverse of ``isTypeOf''. Also, object properties have domains and ranges. Here, ``FeedbackSurvey'' is the range and ``Survey'' is the domain for ``isTypeOf''. Similarly, all the object properties in survey ontology can have inverse, domain and range

\subsection{Data Properties}
Data Properties are used to restrict data entry for particular fields of the tabular data. For survey ontology, data restriction properties restrict the datatype of values of answers for all 9 likert scale questions using Description view of data property hierarchy. The domain and range for these data properties are ``Respondent'' and ``xsd:positiveInteger'' respectively.

\subsection{Individuals}
Individuals are the instances of the classes described for survey ontology. For instance, ``Respondent" class has instances, namely, ``Respondent1", ``Respondent2", ``Respondent3", ...
\section{Conclusion}
The research has potential for further development where the LOD 
(with 5-star rating\footnote{\tt https://www.w3.org/2011/gld/wiki/5\_Star\_Linked\_Data}) 
can be created as a discoverable resource on the web which would enable the citizens to generalize, access and publish their data on the web.

%
% ---- Bibliography ----
%
% BibTeX users should specify bibliography style 'splncs04'.
% References will then be sorted and formatted in the correct style.
%
\bibliographystyle{splncs04}
\bibliography{parita}
\end{document}
