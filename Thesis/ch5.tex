%!TEX root = parita-msc.tex

\chapter{Conclusion and Future Work}
\label{chap:conclusion}
\begin{doublespace}
This Chapter focuses on the prime offerings of this research work where one of the sections includes the conclusions drawn from the work done and the other discusses future work that can be done to succeed in this research.
\section{Conclusion}
\par This thesis work describes a new attempt on developing a linked open data instrument for feedback surveys. It focuses on the issues encountered and the user experience of various tools and references used for the modelling of the instrument and for data visualization. The thesis revolves around some fundamental topics like Semantic Web, Linked Open Data, \ac{rdf}, and \ac{owl} ontologies. It also describes how and why Protégé has been used over other ontology development tools to describe survey vocabularies. The thesis can be considered as a guide for the researchers who are interested in exploring this lesser known aspect of semantic web.
\par The thesis includes a detailed background study on the fundamental topics, the usability of Protégé, and discussion on some tools for linked open data visualizations. Chapter~\ref{chap:ontology} describes the steps involved in building the survey ontology using Protégé, various parts and facilities offered by Protégé, and the visual representation of the ontology toward the end.
\section{Future Work}
%\par A new aspect opens a wide range of future opportunities for the researchers to work upon. 
\par This thesis also offers a number of future development options like building a real-time linked open data application for feedback surveys that 
%obeys 
follows the linked data life cycle. 
%The current research mentions a list of vocabularies which can be defined with detailed properties and published on one of the open platforms. 
Web Scraping (or Web Extraction) is a method of extracting desired data from the resources on the \ac{www} into a separate file for data retrieval and data analysis. Data Extraction is one of the steps in the linked data life cycle that can be explored in future research to facilitate use of data that has already been published to the web with HTML tables.
\par Versioning of the survey ontology is a topic of interest. The ontology vocabularies (available in Appendix~\ref{chap:appendix3}) can be reused for appropriate applications but they may need to be updated over time.
%as the vocabularies are not static. The existing vocabularies might not be specific to a particular purpose. 
Hence, ontology versioning and version management of the survey instrument for this feedback survey,
as well as for other types of surveys is an aspect of this research that can be explored.
\par Developing a data entry form for the feedback survey instrument
that can be defined, in part, directly from the survey ontology would facilitate recording of survey responses collected offline.
%can also be added to the list of future works. A data entry form depending on the data required by the survey instrument can be developed which facilitates entering and recording the responses collected for the survey.

 \end{doublespace}