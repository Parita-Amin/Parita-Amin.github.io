%!TEX root = parita-msc.tex

\begin{doublespace}
\begin{center}
\textbf{Abstract}
\end{center}

The \ac{www}, as defined by the \ac{w3c}
at {\tt https://www.w3.org/WWW/}, 
is ``the universe of network-accessible information, 
the embodiment of human knowledge''. 
With development 
%in study and research 
over the years, 
the global web of linked documents has become the global web of linked data, 
which is popularly known as the Semantic Web.
%\footnote{https://www.w3.org/standards/semanticweb/}.

The Semantic Web is now an established topic of research but there remain many unanswered questions.
\ac{lod}, as the highest standard for linked data, is a worthy goal for researchers and data providers
who seek to contribute to the development of the Semantic Web.
Data that are linked and open facilitate discovery and reuse.
Likewise, vocabularies used to describe the semantic relationships amongst data
that are linked and open facilitate discovery and reuse.
%, to become parts of new data
Yet, it is not always a straightforward matter to take data that may be somehow accessible on a webpage written in \ac{html} and publish it as \ac{lod}.
%and it has not been completely explored yet. 
%In addition to this, open source gives comparatively more development opportunities to all the researchers and developers to discover, build and reuse existing data. 
%The linked open data on the web is designed using linked open vocabularies.
%This thesis describes a model of creating new linked open vocabularies or reusing the existing ones. 
%The thesis also includes the 
Many complimentary technologies are used in handling \ac{lod} on the web,
particularly the \ac{rdf}, the \ac{owl}, and the \ac{sparql}. 
%All these technologies have their own distinct purposes which are further discussed in the thesis.
 
 \marginpar{last paragraph news some attention -- presently commented out}
%Vandenbussche et al.~\cite{pierre2017vocabularies} in 2017 stated 
%that \say{the \ac{lov} initiative gathers and makes visible indicators such as the interconnections between vocabularies and each vocabulary's version history, 
%along with past and current editor (individual or organization)}. 
%This thesis work is in contribution to the same initiative and to Berners-Lee's vision. 
%As a part of this research, 
%a model for survey ontology was developed in order to assist the study which is described in detail in Chapter~\ref{chap:ontology}. 
%There are multiple opportunities for future work like
%creation of new vocabularies required for other types of surveys and more research work.

\end{doublespace}